\begin{figure}

{\scriptsize 
\begin{tabular}{llr@{~~~}r@{~~~}c} 
%\scriptsize
&  &\multicolumn{2}{c}{\textbf{MRE}} & \\ 
  {\textbf{Rank}}& \textbf{China} & \textbf{Med} & \textbf{IQR} & \\\hline 
   
   \rowcolor{gray!20}  1* &      CART\_DE8 &    8 &  1 & \quart{7}{1}{8}{100} \\
  \rowcolor{gray!20}   1* &      CART\_DE2 &    9 &  2 & \quart{8}{2}{9}{100} \\
    2 &      ABEN\_NSGA2 &    12 &  6 & \quart{12}{6}{12}{100} \\
    2 &      CART0 &    15 &  2 & \quart{14}{2}{15}{100} \\
    3 &      ABEN\_DE8 &    25 &  16 & \quart{18}{16}{25}{100} \\
    3 &      ABEN\_DE2 &    28 &  18 & \quart{20}{18}{28}{100} \\
    3 &      ABEN\_RD160 &    30 &  18 & \quart{23}{18}{30}{100} \\
    3 &      ABEN\_RD40 &    34 &  16 & \quart{26}{16}{34}{100} \\
    4 &      CART\_MOEAD &    40 &  10 & \quart{37}{10}{40}{100} \\
    4 &      CART\_NSGA2 &    40 &  17 & \quart{34}{17}{40}{100} \\
    5 &      ABE0 &    54 &  10 & \quart{47}{10}{54}{100} \\
    5 &      CoGEE &    57 &  11 & \quart{52}{11}{57}{100} \\
    6 &      ATLM &    69 &  18 & \quart{69}{18}{76}{100} \\
 \end{tabular}}
 
 \caption{Example of Scott-Knott results.
 MRE scores seen in  the China data set.
 sorted by their median value. 
 Here, {\em smaller} values are {\em better}.
  {\bf Med} is the 50th percentile and {\bf IQR} is the {\em inter-quartile range}; i.e., 75th-25th percentile. 
    Lines with a dot in the middle  shows   median values with the IQR.   
  For the  {\bf Ranks},  {\em smaller} values are  {\em better}.
   Ranks are computed via the Scott-Knot procedure from  TSE’15~\cite{Mittas13}.
    Rows with the same ranks
    are statistically indistinguishable. 
  \colorbox{gray!20}{1*} denotes rows of fastest best-ranked treatments.}\label{eg}
\end{figure}